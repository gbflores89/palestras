%%%%%%%%%%%%%%%%%%%%%%%%%%%%%%%%%%%%%%%%%
% Beamer Presentation
% LaTeX Template
% Version 1.0 (10/11/12)
%
% This template has been downloaded from:
% http://www.LaTeXTemplates.com
%
% License:
% CC BY-NC-SA 3.0 (http://creativecommons.org/licenses/by-nc-sa/3.0/)
%
%%%%%%%%%%%%%%%%%%%%%%%%%%%%%%%%%%%%%%%%%

%----------------------------------------------------------------------------------------
%	PACKAGES AND THEMES
%----------------------------------------------------------------------------------------

%\documentclass{beamer}
\documentclass[aspectratio=169]{beamer}
\mode<presentation> {
% The Beamer class comes with a number of default slide themes
% which change the colors and layouts of slides. Below this is a list
% of all the themes, uncomment each in turn to see what they look like.

%\usetheme{default}
%\usetheme{AnnArbor}
%\usetheme{Antibes}
%\usetheme{Bergen}
%\usetheme{Berkeley}
%\usetheme{Berlin} % possivel
%\usetheme{Boadilla}
%\usetheme{CambridgeUS}
%\usetheme{Copenhagen} % possivel
%\usetheme{Darmstadt} % possivel
%\usetheme{Dresden}
%\usetheme{Frankfurt}
%\usetheme{Goettingen}
%\usetheme{Hannover}
%\usetheme{Ilmenau} % possivel
%\usetheme{JuanLesPins} % possivel
%\usetheme{Luebeck}
%\usetheme{Madrid}
%\usetheme{Malmoe}
%\usetheme{Marburg} % possivel
%\usetheme{Montpellier}
%\usetheme{PaloAlto}
%\usetheme{Pittsburgh}
%\usetheme{Rochester}
%\usetheme{Singapore}
%\usetheme{Szeged} % possivel
\usetheme{Warsaw} % igual o que o professor usa

% As well as themes, the Beamer class has a number of color themes
% for any slide theme. Uncomment each of these in turn to see how it
% changes the colors of your current slide theme.

%\usecolortheme{albatross}
%\usecolortheme{beaver} % possivel
%\usecolortheme{beetle} % possivel, pouco contraste
%\usecolortheme{crane}
%\usecolortheme{dolphin} % possivel
%\usecolortheme{dove}
%\usecolortheme{fly}
%\usecolortheme{lily}
%\usecolortheme{orchid}
%\usecolortheme{rose}
%\usecolortheme{seagull} % P&B
%\usecolortheme{seahorse}
%\usecolortheme{whale}
%\usecolortheme{wolverine}

%\setbeamertemplate{footline} % To remove the footer line in all slides uncomment this line
%\setbeamertemplate{footline}[frame number] % To replace the footer line in all
% slides with a simple slide count uncomment this line

 \setbeamertemplate{navigation symbols}{} % To remove the navigation symbols
% from the bottom of all slides uncomment this line
}

\usepackage{graphicx} % Allows including images
\usepackage{booktabs} % Allows the use of \toprule, \midrule and \bottomrule in tables
\usepackage[brazilian]{babel}
\usepackage[utf8]{inputenc}
\usepackage[T1]{fontenc}
\usepackage{animate}
\usepackage{listings}	% Pacote para linguagens
\let\oldlstlistoflistings\lstlistoflistings
 \renewcommand{\lstlistoflistings}{%
   \begingroup%
   \let\oldnumberline\numberline%
   \renewcommand{\numberline}{\lstlistingname~\oldnumberline}%
   \oldlstlistoflistings%
   \endgroup}
   
 \lstdefinelanguage{EMSO}{
  morekeywords={
		Model,FlowSheet,Estimation,Optimization,
		PARAMETERS,VARIABLES,EQUATIONS,DEVICES,SET,SPECIFY,GUESS,CONNECTIONS,
		INITIAL,OPTIONS,ATTRIBUTES,ESTIMATE,EXPERIMENTS,MINIMIZE,MAXIMIZE,FREE,
		Brief,Unit,DisplayUnit,Default,Upper,Lower,Type,final,Valid,Symbol,PosX,PosY,
		Components,LiquidModel,VapourModel,
		Pallete,Icon,Info,Protected,Hidden,
		TimeStep,TimeEnd,TimeUnit,Dynamic,SparseAlgebra,NLPSolver,NLPSolveNLA,NLASolver,DAESolver,
		File,RelativeAccuracy,AbsoluteAccuracy,EventAccuracy,MaxIterations,GuessFile,InitialFile,
		Statistics,Fit,Parameter,Prediction,Significance,BiLateral,RunTests,NumJac,
		using,for,as,to,if,end,else,switch,switchto,when,case,
		in,out,outer, 
		Integer,Real,Boolean,Text,Plugin,Switcher,CalcObject,
		equal,and,or,true,false,
        time,diff,sum,sumt,prod,prodt,transp,min,max,abs,
		sign,round,sinh,cosh,tanh,coth,atan,
		exp,log,ln,sqrt,sin,cos,tan,asin,acos},
  sensitive = true,
  morecomment=[l]{\#},
  morecomment=[s]{\#*}{*\#},
  morestring=[b]",
  morestring=[b]'
}
\lstset{
  basicstyle=\fontfamily{pcr}\fontseries{m}\selectfont\footnotesize,
  commentstyle=\color[rgb]{0,0.5,0}\itshape,
  keywordstyle=\color{blue}\bfseries,
  stringstyle=\color[rgb]{0.5,0,0.5}\itshape,
  showstringspaces=false,
  numbers=left,
  numberstyle=\fontfamily{pcr}\fontseries{m}\selectfont\tiny,
  numberblanklines=false,
  showlines=false,
  belowskip=\bigskipamount{},
  breaklines=true,
  %stepnumber=2,
  tabsize=6,
  %extendedchars=true,
  %float=h,
  frame=tb
}
\renewcommand{\lstlistingname}{Código}
\renewcommand{\lstlistlistingname}{Lista de Códigos}
  
\setbeamercovered{transparent}

\pgfdeclareimage[width=6cm]{logobig}{img/logo/dequi}
\pgfdeclareimage[width=1.0cm]{logo}{img/logo/dequismall}
\logo{\pgfuseimage{logo}}


  

%----------------------------------------------------------------------------------------
%	TITLE PAGE
%----------------------------------------------------------------------------------------

\title[Simulação na Industria Química]{Simulação na Industria Química.\\
Termodinâmica na indústria e em simuladores }
% The short title appears at the bottom of every slide, the full title is only
% on the title page

\author{Eng. Guilherme Braganholo Flôres} % Your name
\institute[PPGEQ - UFRGS] % Your institution as it will appear on the bottom of
% every slide, may be shorthand to save space
{
UNIVERSIDADE FEDERAL DO RIO GRANDE DO SUL \\
ESCOLA DE ENGENHARIA \\
DEPARTAMENTO DE ENGENHARIA QUÍMICA \\
LAB. VIRTUAL DE PREDIÇÃO DE PROPRIEDADE \\ % Your institution for the title page
\medskip
\textit{gbflores89@gmail.com} % Your email address
} 
\date{\today} % Date, can be changed to a custom date

\begin{document}
\logo{\pgfuseimage{logobig}}
\begin{frame}
\titlepage % Print the title page as the first slide
\end{frame}

\begin{frame}
	\frametitle{Guilherme Braganholo Flôres}
	Engenheiro Químico formado pela UFRGS
	\begin{itemize} 
		\item Universidade Federal do Rio Grande do Sul
	\end{itemize}

	Mestrando em Engenharia Química pelo PPGEQ da UFRGS
	\begin{itemize}
		\item Programa de Pós-Graduação em Engenharia Química
	\end{itemize}
 	\pause
	Área de pesquisa:
	\begin{itemize}
		\item LVPP
		\begin{itemize}
			\item Laboratório Virtual de Predição de Propriedades
		\end{itemize}
	\item Modelagem industrial 
	\item Modelos termodinâmicos
	\end{itemize}
\end{frame}


\begin{frame}
\frametitle{Sumário} % Table of contents slide, comment this block out to
% remove it
\tableofcontents % Throughout your presentation, if you choose to use
% \section{} and \subsection{} commands, these will automatically be printed on
% this slide as an overview of your presentation
\end{frame}

%------------------------------------------------------------------------------
%	PRESENTATION SLIDES
%------------------------------------------------------------------------------
\logo{\pgfuseimage{logo}}
\section{Introdução}
\begin{frame}
	\frametitle{O que devemos ter em mente}
	\begin{itemize}
		\item O que é um simulador de processos?
		\item Para que serve um simulador de processo?
		\item Quais são as vantagens do uso de simuladores?
		\item Como são utilizados os simuladores dentro das industrias?
		\item Quais os simuladores disponives? 
		\item Como um simulador funciona?
	\end{itemize}
\end{frame}

\subsection{Simuladores}

\begin{frame}
	\frametitle{O que são simuladores}
	\begin{columns}[c]
		\column{.5\textwidth}
		\begin{itemize}
			\item Qualquer programa que transforme um ''coisa'' real 
			em um (ou vários) sistema(s) de equações.
			\note<1>{Simuladores são todos os programas que podemos transformar sistemas
			reais em modelos para “prever o futuro”.\\ .\\
			Por exemplo, os simuladores de carros, simuladores de aviões,
			por que não jogos de carros e aviões (?), simuladores de
			investimento, simuladores empresariais e assim por diante..\\ .\\
			Um exemplo, que mais adiante vocês irão se deparar, é o VPL, uma das simulações
			mais simples que existe, serve para prever como que eu vou pagar um
			investimento, em quanto tempo posso ter o meu dinheiro de volta.}
			\begin{itemize}
				\item <2->O sistema pode ser resolvido por cálculos feitos à
				mão ou programas computacionais
			\end{itemize}
		\end{itemize}
		\note<2>{Outro exemplo extremo são os simuladores de treinamento da NASA,
		muito mais complexos que qualquer simulador de processo, mas o investimento que a NASA fez
		para esses simuladores são pagos pela segurança que eles trazem.\\.\\
		Geralmente o piloto da nave espacial nunca guiou uma de verdade, mas passou horas e horas
		treinando em um simulador, e provavelmente já morreu diversas vezes de forma
		fictícia, é logico.}
		\column{.5\textwidth} % Right column and width
			\includegraphics[width=0.95\textwidth]
			{img/Molecular_simulation_process.png}
	\end{columns}
\end{frame}

\begin{frame}
	\frametitle{O que são simulador de processo?}
	\begin{columns}[c] 
		\column{.3\textwidth}
		Em simuladores de processo a ``coisa'' passa a ser a planta industrial, ou
		parte dela.
		\column{.7\textwidth}
		\begin{center}
			\includegraphics[width=0.5\textwidth]
				{img/RefineryFlow.png}
		\end{center}
	\end{columns}
\end{frame}

\begin{frame}
	\frametitle{Principais motivos para o uso de um simulador}
	Projeto de uma planta química\\
	\note<1>{Quando o cálculo feito à mão é muito custoso, ou seja, podemos dar um
	chute inicial para o cálculo de uma coluna de destilação, por exemplo, mas o
	cálculo rigoroso, será feito apenas via simulador.\\.\\}

	Otimização de uma planta química\\
	\note<1>{Podemos inclusive, otimizar uma planta ainda em estado de
	desenvolvimento.\\.\\}
	
	Fazer diversos experimentos sem custo financeiro ou com custo mínimo!
	\note<1>{E como sabemos, para qualquer otimização, devemos conhecer o
	comportamneto da planta, ou seja, quais são os limites operacionais, e a
	matriz de causa e efeito, se eu mexer neste parâmetro, qual o resultado da
	planta?? e por assim em diante.\\.\\}
	\begin{itemize}
		\item Experimentos que industrialmente são muito demorados podem ser feitos em
		questões de minutos
			\begin{itemize}
				\item Colunas de destilação podem demorar dias para entrar
				em estado estacionário
		\end{itemize}
	\end{itemize}
	\pause
	Segurança
	\begin{itemize}
		\item Pode-se ultrapassar, sem risco os limites operacionais dos equipamentos
	\end{itemize}
\end{frame}


\begin{frame}
	\frametitle{Simuladores mais conhecidos}
	\begin{itemize}
		\item Aspen Technology --> Aspen Plus
		\item Aspen Technology --> Aspen HYSYS
		\item PSE Ltd --> gPROMS
		\item SimSci -->PRO/II \pause 
		\note<1>{Os simuladores mais conhecidos na indústria e na academia são
		estes:\\.\\ O Aspen Plus (caro, mas muito bom) e o Aspen HYSYS... \\
		O HYSYS foi comprado pela Aspen poucos anos atrás e como já era muito
		conhecido, ficou o nome HYSYS, são simuladores de blocos ou seja, coloca os
		equipamentos e o simulador resolve um por vez, até que finalmente o sistema
		inteiro converge.\\.\\
		gPROMS eu sinceramente não conheço, mas sempre ouço falar
		PRO/II segue a mesma linha do Aspen Plus e HYSYS}
		
		\item \textbf{ALSOC Project --> EMSO}
		\item \textbf{VRTech --> iiSE Simulator}
		\note<2>{ E estes dois últimos, o EMSO e o iiSE, são os simuladores que eu
		mais trabalho, a vantagem destes simuladores é que eles são orientados a
		equações, ou seja, transformam o fluxograma de processo em um sistema de
		equações, onde o tamanho do sistema vai variar dependendo da complexidade
		dele.}
	\end{itemize}
\end{frame}

\begin{frame}
	\frametitle{Tipo de simuladores}
	\begin{columns}[t] 
		\column{.5\textwidth}
			\textbf{Simuladores em blocos.}\\
			Aspen Plus, por exemplo.\\
			\begin{itemize}
				\item A simulação é feita por etapas, um bloco por vez.
				\begin{itemize}
					\item Vários sistemas de equações menores resolvidos separadamente
					\item O resultado de um equipamento depende diretamente do 
				equipamento calculado anteriormente.
				\end{itemize}
				\item Apresenta mais problemas com reciclos
			\end{itemize}
		\column{.5\textwidth}
			\textbf{Simuladores orientado a equações.}\\
			EMSO e iiSE, por exemplo.\\
			\begin{itemize}
				\item A simulação é feita em uma unica etapa.
				\begin{itemize}
					\item A simulação inteira é transformado em um unico sistema de equações
					\item A solução dos equipamentos é feita de forma simultanea para
					todos.
				\end{itemize}
			\end{itemize}
	\end{columns}
	\note{Bom, temos dois tipos de simuladores, os simuladores de blocos e os
	orientados a equações\\
	Os simuladores de blocos, como o Aspem, transformam o fluxograma de processo
	em vários sistemas de equações, um para cada equipamento e resolve o
	problema inteiro em etapas.\\.\\
	Por outro lado os simuladores orinetados a equações transformam o fluxograma
	como um todo em um unico sistema de equações e resolve tudo de uma vez só}
\end{frame}

\begin{frame}
	\frametitle{Simuladores de blocos}
	Fazer aqui uma figura com vários blocos para demonstrar o simulador funcionando
\end{frame}

{
\begingroup
\setbeamertemplate{headline}{}
\addtobeamertemplate{frametitle}{\vspace*{-0.9\baselineskip}}{}
\setbeamertemplate{headline}{}
\begin{frame}
	\frametitle{EMSO}
	\begin{center}
		\includegraphics[width=0.9\textwidth]{img/EMSO_1.PNG}
	\end{center}
	\note{Funciona mais ou menos assim, o usuário abre um novo arquivo e vai
	montando a simulação, vai fazendo aos poucos, cada vez que o usuário manda o
	comando para executar a simulação, o simulador automaticamente transforma todos
	os equipamentos (blocos) em um único sistema de equações, pois cada bloco é
	composto por uma serie de equações \\
	No caso este exemplo é um pequeno pedaço de uma plataforma de pertróleo}
\end{frame}

\begin{frame}
	\frametitle{EMSO}
	\begin{center}
		\includegraphics[width=0.9\textwidth]{img/EMSO_2.PNG}
	\end{center}
	\note{Somente depois de tudo transformado em equações, que ai sim o simulador
	começar a calcular o sistema.\\.\\
	Bom ai vai de cada usuário saber o que deseja simular, nesse sentido eu digo
	que da pra simular tudo que acontece na indústria química, desde que utilize o
	simulador adequado }
\end{frame}

\endgroup
}
\subsection{Modelos}

\begin{frame}
	Ok já sabemos o que são simuladores, mas como eles funcionam?\\
	\begin{itemize}
	  \item Cada equipamento do simulador é um \code{modelo}
	\end{itemize} \pause
	Como são estes \code{modelos}?
\end{frame}

{
\begingroup
\setbeamertemplate{headline}{}
\addtobeamertemplate{frametitle}{\vspace*{-0.9\baselineskip}}{}
\setbeamertemplate{headline}{}
\begin{frame}
	\frametitle{Modelo EMSO}
	\footnotesize
		\begin{columns}[t] 
			\column{.5\textwidth}
				 \lstinputlisting[firstline=3, lastline=14, 
				 numbers=none, language=EMSO]
				 {img/pump.mso}
			\column{.5\textwidth}
				 \lstinputlisting[firstline=15, lastline=33, 
				 numbers=none, language=EMSO]
				 {img/pump.mso}
		\end{columns}
	\note{Todos os simuladores tem um código assim implementado, inclusive em
	simuladores de blocos, óbvio que podemos ter implementações mais ou menos
	rigorosas, esse exemplo é uma bomba simplificada, se bem que ainda podemos
	simplificar mais ainda.}
\end{frame}

\begin{frame}
 	\frametitle{Modelo simplificado}
%  	\framesubtitle{Não façam isso em casa!!}  
	\begin{columns}[c] 
		\column{.5\textwidth}
			 \lstinputlisting[firstline=3, lastline=17, 
			 numbers=none, language=EMSO] 
			 {img/pump2.mso} 
		\note<3>{Por favor, não façam isso em casa, e nem quando estiverem trabalhando
		com um modelo sério\\.\\}
		\pause
		\column{.5\textwidth}
			\begin{itemize}
				\item Por que alguém faz um modelo assim??
				\item Ok mas para que serve um modelo deste tipo??
			\end{itemize}
			\pause
			\begin{block}{}
				\begin{itemize}
					\item Simples, a bomba não é importante na minha simulação, tanto faz se
					esqunta o fluido ou não.
					\item Só preciso dela para um incremento de pressão.
				\end{itemize}
				\note<3>{Isso nos tras outra questão, o quanto certo está o nosso modelo}
			\end{block}
	\end{columns}
\end{frame}
\endgroup
}


\begin{frame}
	\frametitle{Modelos}
	\begin{columns}[c]
		\column{.3\textwidth}
	\begin{itemize}
	  \item O que eu quero simular? \pause
	  \note<3>{De nada adianta eu criar um modelo se eu não sei o que eu quero
	  fazer com ele, isso é mais ou menos como: Ta indo pra onde? Não sei, mas to
	  indo\\.\\}
	  \item Os modelos estão certos? \pause
	  \note<3>{E agora temos um problema.\\.\\}
	  \begin{itemize}
		\item Não!!!
		\item <alert@3> Todos os modelos estão ``errados''.
	  \end{itemize}
	  \note<3>{O que acontece é que todos os modelos estão errados, não importa o
	  quanto esforço seja empregado para ele, ele vai estar errado. A parte boa da
	  hitória é que mesmo errado, o modelo vai nos ajudar}
	\end{itemize}
		\column{.7\textwidth}
		\includegraphics[width=0.9\textwidth]{img/blocos.png}
	\end{columns}
\end{frame}

\begin{frame}
	\frametitle{Confiabilidade dos modelos}
	Mas se todos os modelos estão errados,
	como podemos confiar neles?\\
	\begin{itemize}
	  \item Depende das simplificações que podemos admitir \pause
	  \item Em um modelo rigoroso, por exemplo as simplificações são minimas e
	  tudo que é possivel de ser calculado será calulado!
	\end{itemize}
	\note<2>{Sim os modelos estão errados, mas depende do quanto errado podemos
	admitir, em um modelo simplificado, como eu mostrei anteriormente sabemos que
	foram feitas varias suposições como}
\end{frame}

\begin{frame}
	\frametitle{Confiabilidade dos modelos}
	\begin{columns}[c] 
	\column{.5\textwidth}
		Dependo das simplificações que forem feitas o modelo vai servir para o
		nosso proposito.\\
		\begin{itemize}
			\item Não podemos simplificar no caso de uma bomba, pro exemplo \\
 				\begin{equation*}
				P_{in}=P_{out}
 				\end{equation*}\\
			\item <2->Mas em um modelo simplificado podemos desconsiderar a curva da
			bomba
		\end{itemize}
		\column{.5\textwidth}
		\begin{center}
			\includegraphics<2->[width=0.65\textwidth]{img/curva_bomba.jpg}
		\end{center}
	\end{columns}
\end{frame}

\section{Modelos Termodinâmicos}

\begin{frame}
	\frametitle{O que já podemos decidir até o momento?}
	\begin{itemize}
		\item Já definimos o que queremos simular.
		\item Já definimos o simulador.
		\item Já definimos o qunto o nosso modelo pode ser simplificado.
	\end{itemize}
	\pause
	Como calculamos as propriedades termodinâmicas que a simulação irá precisar?
\end{frame}

\begin{frame}
	\frametitle{Tipos de modelos termodinamicos}
	\begin{columns}[t] 
		\column{.5\textwidth}
			\textbf{Componentes puros}\\
			\begin{itemize}
				\item Equacões de estado.
			\end{itemize}
		\column{.5\textwidth}
			\textbf{Misturas}\\
			\begin{itemize}
				\item Equações de estado combinada à regras de misturas.
				\item Modelos de $G^E$ \\ 
				Também conhecidos como modelos de $\gamma$
				\item Modelos de $G^E$ combinado com equações de estado via regra de
				mistura.
			\end{itemize}
	\end{columns}
\end{frame}

\subsection{Equações de Estado}

\begin{frame}
	\frametitle{Equações de Estado (Equations of State - EOS)}
	\begin{itemize}
		\item De alguma forma relacionam as propriedades $P$, $v$ e $T$ ou
		outras propriedades não mensuráveis
		\item Exemplos: gás ideal, cúbicas (PR, SRK, ...), tipo Virial, SAFT e
		PC-SAFT
		\item Partindo de uma EOS e do $Cp^{GI}$ é \textbf{possível calcular todas
		as demais propriedades}: $u$, $h$, $s$, ...
		\item São desenvolvidas para substâncias puras (a maioria)
	\end{itemize}
\end{frame}

\begin{frame}
	\frametitle{O Gás Ideal}
	É a equação de estado mais comum
	\begin{equation*}
	P=\frac{RT}{v}
	\end{equation*}
	\begin{itemize}
	  \item Baseada nas considerações de que:
	  \begin{enumerate}
	    \item As moléculas do gás não ocupam nenhum volume
	    \item As moléculas do gás não exercem nenhuma força
		intermolecular
	  \end{enumerate}
	\end{itemize}
	Entretanto apenas pode representar a fase gás
\end{frame}

\begin{frame}
	\frametitle{Equações cúbicas - Equação de van der Waals}
	\begin{columns}[c] 
	\column{.7\textwidth}
		\begin{itemize}
		\item Proposta por van der Waals em 1873
		\item Considera as moléculas como esferas \\
		rígidas para representar as forças\\
		repulsivas
		\item Forças de van der Waals descrevem as\\
		forças atrativas
		\item Depende de dois parâmetros: $a$ e $b$
		\end{itemize}
	\column{.3\textwidth}
	\begin{center}
		\includegraphics[width=0.8\textwidth]{img/waals.jpg}
	\end{center}
	\end{columns}
	\begin{columns}
		\column{.75\textwidth}
			\begin{exampleblock}{Johannes Diderik van der Waals, Nobel em Física - 1910}
				... for his studies of the physical state of liquids and gases..
			\end{exampleblock}
	\end{columns}
\end{frame}

\begin{frame}
	\frametitle{Equações cúbicas - Equação de van der Waals}
	\begin{columns}
		\column{.75\textwidth}
			\begin{exampleblock}{Correção devido ao tamanho das moléculas:}
				volume molar não ocupado pelas moléculas, $v-b$
			\end{exampleblock}
	\end{columns}
	\begin{columns}
		\column{.75\textwidth}
			\begin{exampleblock}{Correção devido às forças intermoleculares
			atrativas:}
			forças atrativas proporcionais a ${}^{1}/{}_{r^6}$ ou
			${}^{1}/{}_{v^2}$
			\end{exampleblock}
	\end{columns}
	\begin{itemize}
		\item Forma final
		\begin{equation*}
		P=\frac{RT}{v-b}-\frac{a}{v^2}
		\end{equation*}
		\item Pode ser re-escrita na forma cúbica como:
		\begin{equation*}
		Pv^3-\left(RT+Pb\right)v^2+av-ab=0
		\end{equation*}
	\end{itemize}
\end{frame}

\begin{frame}
	\frametitle{Equações cúbicas - Equação de van der Waals}
	\begin{itemize}
		\item Os parâmetros $a$ e $b$ da equação de vdW podem ser obtidos a
		partir do ponto crítico: 
		\begin{equation*}
		\left(\frac{\partial P}{\partial v}\right)_{T_r} = \left(\frac{\partial^2
		P}{\partial v^2}\right)_{T_r}=0
		\end{equation*}
		\begin{equation*}
		\Downarrow
		\end{equation*}
		\begin{equation*}
		a=\frac{27}{64}\frac{\left(RT_c\right)^2}{P_c} \qquad
		b=\frac{RT_c}{8P_c} 
		\end{equation*}
	\end{itemize}
\end{frame}

\begin{frame}
	\frametitle{Algumas Equações Cúbicas}
	Outras equações cúbicas:
	\begin{equation*}
		P= \frac{RT}{v-b}-\frac{a(T)}{(v-\epsilon b)(v+\epsilon b)}
	\end{equation*}
	\begin{equation*}
		a= \frac{\Psi \alpha(T_r) R^2 {T_c}^2}{P_c} \quad b= \Omega \frac{RT_c}{P_c}
		\quad T_r = \frac{T}{T_c} \quad P= \frac{P}{P_c}
	\end{equation*}
	\begin{table}
		\begin{tabular}{cc|ccccc}
		\hline
		{EoS} & Ano & {$\alpha(T_r)$} & {$\sigma$}& {$\epsilon$}&
		{$\Omega$}& {$\Psi$} \\
		\hline
		vdW  & 1873& $1$ & 0 & 0 & $1/8$ & $27/64$\\
		RK & 1949 & $1/\sqrt{T_r}$ & 1 & 0 & 0.08664& 0.42748\\
		SRK & 1972 & $\alpha_{SRK}$ & 1 & 0 & 0.08664& 0.42748\\
		PR & 1976 & $\alpha_{PR}$ & $1+\sqrt{2}$ & $1-\sqrt{2}$ & 0.07780 & 0.45724\\
		\hline
		\multicolumn{7}{l}{$\alpha_{SRK}=\left[1+\left(0.48+1.574w-0.176w^2\right)
		\left(1-\sqrt{T_r}\right) \right]^2$}\\
		\multicolumn{7}{l}{$\alpha_{PR}=\left[1+\left(0.37464+1.54226w-0.2699w^2\right)
		\left(1-\sqrt{T_r}\right) \right]^2$}\\
		\hline
		\end{tabular}
	\end{table}
\end{frame}

\begin{frame}
	\frametitle{Comportamento $PvT$ de acetonitrila calculado com PR}
	\begin{center}
		\includegraphics[width=0.75\textwidth]{img/Acetonitrila.png} 
	\end{center}
\end{frame}

\begin{frame}
	\frametitle{Comportamento $PvT$ de acetonitrila calculado com GI}
	Se possível mostrar com GI igual a essa img
	\begin{center}
		\includegraphics[width=0.6\textwidth]{img/Acetonitrila.png} 
	\end{center}
\end{frame}

\subsection{Regras de Mistura}

\begin{frame}
	\frametitle{Regras de Mistura}
	\begin{columns}[c]
	\column{.5\textwidth}
		\begin{itemize}
			\item A grande maioria das equações de estado são desenvolvidas para
			\textbf{substâncias puras}!
			\item Para misturarmos substâncias devemos utilizar \textbf{regras de
			misturas}
			\begin{itemize}
				\item Regras de misturas ``combinam as propriedades das substâncias puras em
				um único parâmetro para que possamos utilizar nas EoS''	
			\end{itemize}
		\end{itemize}
	\column{.5\textwidth}
		\begin{itemize}
			\item A primeira regra de mistura veio pela necessidade da primeira EoS
			\begin{block}{Regra de mistura de van der Waals}
			\begin{equation*}
			a = \sum_i\sum_j{y_iy_ja_{ij}} \quad a_{ij}= \sqrt{a_i a_j}
			\end{equation*}
			\begin{equation*}
			b = \sum_i{b_iy_i}
			\end{equation*}
			\end{block}
		\end{itemize}
	\end{columns}
\end{frame}

\begin{frame}
	\frametitle{Outras regras de mistura}
	Comentar sobre a as outras regras de misturas
\end{frame}

\subsection{Modelos de $G^E$}

\begin{frame}
	\frametitle{Modelos de $G^E$}
	\begin{columns}[c]
		\column{.4\textwidth}
		Também conhecidos como:
		\begin{itemize}
			\item Modelos de coeficiente de atividade (da fase líquida)
			\begin{itemize}
				\item Modelos de $\gamma$
			\end{itemize}
		\end{itemize}
		São modelos para representar a não idealidade da fase líquida
		\begin{itemize}
			\item Em sua principal formulação considera o vapor como \\
			\textbf{gás ideal}
		\end{itemize}
		\column{.6\textwidth}
		\includegraphics[width=1.0\textwidth]{img/VLE_ideal.png}
	\end{columns}
\end{frame}

\begin{frame}
	\frametitle{Lei de Raoult}
\end{frame}












 
 
 
 
 
 
 
 
 
 
 
 
 
 
 
 
 
 
 
 
 
 
 
 
 
 
 
 
 
 
 
 
 
 
 
 
\begin{frame}
	\frametitle{}
\end{frame}
 
 
 
\begin{frame}
\frametitle{Teste}
\framesubtitle{The proof uses \textit{reductio ad absurdum}.}
\begin{theorem}
There is no largest prime number.
\end{theorem}
\begin{proof}
\begin{enumerate}
\item<1-| alert@1> Suppose $p$ were the largest prime number.
\item<2-> Let $q$ be the product of the first $p$ numbers.
\item<3-> Then $q+1$ is not divisible by any of them.
\item<1-> But $q + 1$ is greater than $1$, thus divisible by some prime
number not in the first $p$ numbers.\qedhere
\end{enumerate}
\end{proof}
\begin{itemize}
\item 2 is prime (two divisors: 1 and 2).
\pause
\item 3 is prime (two divisors: 1 and 3).
\pause
\item 4 is not prime (\alert{three} divisors: 1, 2, and 4).
\end{itemize}
\end{frame}
 %------------------------------------------------

\begin{frame}
\frametitle{Bullet Points}
\begin{itemize}
\item Lorem ipsum dolor sit amet, consectetur adipiscing elit
\item Aliquam blandit faucibus nisi, sit amet dapibus enim tempus eu
\item Nulla commodo, erat quis gravida posuere, elit lacus lobortis est, quis porttitor odio mauris at libero
\item Nam cursus est eget velit posuere pellentesque
\item Vestibulum faucibus velit a augue condimentum quis convallis nulla gravida
\end{itemize}
\end{frame}

%------------------------------------------------

\begin{frame}
\frametitle{Blocks of Highlighted Text}
\begin{block}{Block 1}
Lorem ipsum dolor sit amet, consectetur adipiscing elit. Integer lectus nisl, ultricies in feugiat rutrum, porttitor sit amet augue. Aliquam ut tortor mauris. Sed volutpat ante purus, quis accumsan dolor.
\end{block}

\begin{block}{Block 2}
Pellentesque sed tellus purus. Class aptent taciti sociosqu ad litora torquent per conubia nostra, per inceptos himenaeos. Vestibulum quis magna at risus dictum tempor eu vitae velit.
\end{block}

\begin{block}{Block 3}
Suspendisse tincidunt sagittis gravida. Curabitur condimentum, enim sed venenatis rutrum, ipsum neque consectetur orci, sed blandit justo nisi ac lacus.
\end{block}
\end{frame}

%------------------------------------------------

\begin{frame}
\frametitle{Multiple Columns}
\begin{columns}[c] % The "c" option specifies centered vertical alignment while the "t" option is used for top vertical alignment

\column{.45\textwidth} % Left column and width
\textbf{Heading}
\begin{enumerate}
\item Statement
\item Explanation
\item Example
\end{enumerate}

\column{.5\textwidth} % Right column and width
Lorem ipsum dolor sit amet, consectetur adipiscing elit. Integer lectus nisl, ultricies in feugiat rutrum, porttitor sit amet augue. Aliquam ut tortor mauris. Sed volutpat ante purus, quis accumsan dolor.

\end{columns}
\end{frame}

%------------------------------------------------
\section{Second Section}
%------------------------------------------------

\begin{frame}
\frametitle{Table}
\begin{table}
\begin{tabular}{l l l}
\toprule
\textbf{Treatments} & \textbf{Response 1} & \textbf{Response 2}\\
\midrule
Treatment 1 & 0.0003262 & 0.562 \\
Treatment 2 & 0.0015681 & 0.910 \\
Treatment 3 & 0.0009271 & 0.296 \\
\bottomrule
\end{tabular}
\caption{Table caption}
\end{table}
\end{frame}

%------------------------------------------------

\begin{frame}
\frametitle{Theorem}
\begin{theorem}[Mass--energy equivalence]
$E = mc^2$
\end{theorem}
\end{frame}

%------------------------------------------------

\begin{frame}[fragile] % Need to use the fragile option when verbatim is used in the slide
\frametitle{Verbatim}
\begin{example}[Theorem Slide Code]
\begin{verbatim}
\begin{frame}
\frametitle{Theorem}
\begin{theorem}[Mass--energy equivalence]
$E = mc^2$
\end{theorem}
\end{frame}\end{verbatim}
\end{example}
\end{frame}

%------------------------------------------------

\begin{frame}
\frametitle{Figure}
Uncomment the code on this slide to include your own image from the same directory as the template .TeX file.
%\begin{figure}
%\includegraphics[width=0.8\linewidth]{test}
%\end{figure}
\end{frame}

%------------------------------------------------

\begin{frame}[fragile] % Need to use the fragile option when verbatim is used in the slide
\frametitle{Citation}
An example of the \verb|\cite| command to cite within the presentation:\\~

This statement requires citation \cite{p1}.
\end{frame}

%------------------------------------------------

\begin{frame}
\frametitle{References}
\footnotesize{
\begin{thebibliography}{99} % Beamer does not support BibTeX so references must be inserted manually as below
\bibitem[Smith, 2012]{p1} John Smith (2012)
\newblock Title of the publication
\newblock \emph{Journal Name} 12(3), 45 -- 678.
\end{thebibliography}
} 
\end{frame}

%------------------------------------------------

\begin{frame}
\Huge{\centerline{The End}}
\end{frame}

%----------------------------------------------------------------------------------------

\end{document} 
